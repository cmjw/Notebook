\chapter{Instruction Set Architecture}

\section{Introduction}

The Instruction Set Architecture (ISA) defines the behavior and format of a machine-level program. The ISA defines the processor state, instruction format, and effects an instruction has on the processor state. 

Common architectures include:
\begin{itemize}
	\item IA32: Intel Architecture (32-bit)
	\item x86: 64-bit extension to IA32 (x86\_64)
	\item AMD
\end{itemize}

\section{CISC Architectures}

\subsection{Complex Instruction Set Computer (CISC)}

\textbf{CISC} architectures have a larger set of instructions which are more complex, and may perform multiple internal operations. These architectures can typically complete a task with fewer instructions, at the cost of complexity at the hardware level. This results in higher potential performance.

\subsection{x86 (Intel)}



\section{RISC Architectures}

\subsection{Reduced Instruction Set Computer (RISC)}

\textbf{RISC} architectures have fewer, simpler instructions, each of which performs a single operation. These architectures are usually less complex, perform well using pipelining, and require less silicon at a hardware level. This approach saves power and cost. RISC processors are often used in embedded systems.

\section{References}

\begin{itemize}
	\item https://www.sciencedirect.com/book/9780128007266/power-and-performance
	\item https://imada.sdu.dk/u/kslarsen/dm546/Material/IntelnATT.htm
	\item https://www.cs.umd.edu/meesh/411/CA-online/chapter/instructionset-
	architecture/index.html
	\item https://www.windriver.com/solutions/learning/leading-processor-architectures
\end{itemize}