\chapter{Boolean Algebra \& Logic Gates}

Boolean algebra is a branch of mathematics that deals with logical operations on binary variables.

\section{Rules and Theorems}

Associative, demorgan, blah blah

\section{Basic Logic Gates}

\subsection{AND}

\begin{figure}[h!]
	\includegraphics{./img/and.png}
\end{figure}

\begin{tabular}{c c c}
	\hline
	\textbf{x} & \textbf{y} & \textbf{x \& y} \\ 
	\hline
	0 & 0 & 0 \\
	0 & 1 & 0 \\
	1 & 0 & 0 \\
	1 & 1 & 1 \\
	\hline 
\end{tabular} \\

\subsection{OR}

\begin{figure}[h!]
	\includegraphics{./img/or.png}
\end{figure}


\begin{tabular}{c c c}
	\hline
	\textbf{x} & \textbf{y} & \textbf{x + y} \\ 
	\hline
	0 & 0 & 0 \\
	0 & 1 & 1 \\
	1 & 0 & 1 \\
	1 & 1 & 1 \\
	\hline 
\end{tabular} \\

\subsection{NOT}

\begin{figure}[h!]
	\includegraphics{./img/not.png}
\end{figure}


\begin{tabular}{c c}
	\hline
	\textbf{x} & \textbf{¬x} \\ 
	\hline
	0 & 1  \\
	1 & 0  \\
	\hline 
\end{tabular} \\

\section{Latches}

\subsection{SR Latch}

An SR (Set-Reset) Latch has two inputs, S and R (Set and Reset). The S input sets the output to 1, and the R input resets the output to 0. An SR latch's output behavior is undefined when both the S and R inputs are at 1.

\subsection{Gated SR Latch}

\subsection{D Latch}

\section{References}

\begin{itemize}
	\item https://bob.cs.sonoma.edu/IntroCompOrg-x64/bookch4.html
\end{itemize}